\documentclass[12pt,a4paper]{article}
\usepackage[utf8]{inputenc}
\usepackage[english]{babel}
\usepackage{glossaries}

% Define acronyms in glossary
\newacronym{ml}{ML}{Machine Learning}
\newacronym{nlp}{NLP}{Natural Language Processing}
\newacronym{gpu}{GPU}{Graphics Processing Unit}

\makeglossaries

\title{Sample Research Paper}
\author{Test Author}
\date{\today}

\begin{document}

\maketitle

\section{Introduction}

Artificial Intelligence (AI) has revolutionized many fields. Machine Learning is a subset of AI that enables systems to learn. The field of Natural Language Processing (NLP) has seen significant advances.

Modern AI systems often require powerful Graphics Processing Units for training. Deep Learning (DL) techniques have become increasingly popular. Computer Vision (CV) applications are widespread in autonomous vehicles.

\section{Related Work}

Previous work in AI has shown promising results. ML algorithms have been applied successfully in various domains. The use of GPU acceleration has made training faster.

Some researchers focus on Artificial Intelligence (AI) applications in healthcare. Others work on Machine Learning for financial prediction. Natural Language Processing techniques are used in chatbots.

\section{Methodology}

Our approach uses Deep Learning methods combined with Computer Vision techniques. We implemented our solution using Graphics Processing Unit acceleration for faster training.

The Convolutional Neural Network (CNN) architecture was chosen for its effectiveness. We also experimented with Recurrent Neural Networks (RNN) for sequence processing.

\section{Results}

The AI system achieved high accuracy on the test dataset. ML performance metrics showed significant improvement over baseline methods. Our NLP model outperformed existing approaches.

\section{Conclusion}

This work demonstrates the potential of AI in solving complex problems. Future work will explore more advanced ML techniques and improved GPU utilization.

\end{document}